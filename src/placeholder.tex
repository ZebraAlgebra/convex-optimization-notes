\chapter{Placeholder Chapter}
\label{chap:placeholder}

\paragraph{}This is a placeholder chapter, used for explaining some setups of this projects.

\section{Building the Document}

\paragraph{}To build the document, you will need \code{latexmk}, \code{pdflatex}, \code{biber} installed, and run \code{latexmk}; the document should build according to the settings in \filename{.latexmkrc}.

\section{Directory Structure}
\label{sect:dir-struct}

\begin{itemize}
	\item Git-related:
	      \begin{itemize}
		      \item \filename{README.md} holds the description of the repo
		      \item \filename{.gitignore} holds things to ignore when doing \code{git add .}
	      \end{itemize}
	\item Directories:
	      \begin{itemize}
		      \item \filename{bib/} directory holding bibliographical entries
		      \item \filename{figures/} directory holding figures to be included in the document
		      \item \filename{src/} directory holding tex files
		      \item \filename{sty/} directory holding sty files, including packages, package settings, definitions of custom environments, commands
	      \end{itemize}
	\item \filename{.latexmkrc} an rc, init file for controling how \code{latexmk} builds the document
	\item \filename{.main.tex} entry point for build
	\item Files ignored by \filename{.gitignore}:
	      \begin{itemize}
		      \item \filename{*.pdf} this is used to ignore the target build pdf; the built pdf name is given by the \code{\$jobname} param in \code{.latexmkrc}
		      \item \filename{*.synctex.gz} artifact when building pdf
		      \item \filename{*.log} this is used to ignore \filename{indent.log}, an artifact generated by the formatter \code{latexindent}
		      \item \filename{logs/} directory holding other artifacts generated by \code{latexmk}, specified by the \code{\$aux\_dir} param in \code{.latexmkrc}
	      \end{itemize}
\end{itemize}

\section{Some Predefined Commands}
\label{sect:cmnds}

\begin{itemize}
	\item Colored texts: one can use \code{code, filename, datatext} to make some text in \code{texttt} style with certain color, e.g. \code{python3}, \filename{main.tex}, \datatext{animals}.
	\item Environments setup for \code{amstheorem}:
	      \begin{center}
		      \begin{tabular}{|l|l|l|}
			      \hline
			      \textbf{Name} & \textbf{Description} & \textbf{Numbering} \\\hline
			      \code{defn}   & {Definition}         & Yes                \\\hline
			      \code{lemm}   & {Lemma}              & Yes                \\\hline
			      \code{coro}   & {Corollary}          & Yes                \\\hline
			      \code{thrm}   & {Theorem}            & Yes                \\\hline
			      \code{prop}   & {Proposition}        & Yes                \\\hline
			      \code{exmp}   & {Example}            & Yes                \\\hline
			      \code{rmrk}   & {Remark}             & Yes                \\\hline
			      \code{asmp}   & {Assumption}         & Yes                \\\hline
			      \code{conv}   & {Convention}         & Yes                \\\hline
			      \code{defn*}  & {Definition}         & No                 \\\hline
			      \code{lemm*}  & {Lemma}              & No                 \\\hline
			      \code{coro*}  & {Corollary}          & No                 \\\hline
			      \code{thrm*}  & {Theorem}            & No                 \\\hline
			      \code{prop*}  & {Proposition}        & No                 \\\hline
			      \code{exmp*}  & {Example}            & No                 \\\hline
			      \code{rmrk*}  & {Remark}             & No                 \\\hline
			      \code{asmp*}  & {Assumption}         & No                 \\\hline
			      \code{conv*}  & {Convention}         & No                 \\\hline
			      \code{proof}  & {Proof}              & No                 \\\hline
		      \end{tabular}
	      \end{center}
	      You can add more in \filename{sty/}. For environments supporting numbering, you can add label by the \code{label} command, and reference it later (see \Cref{sect:referencing}).
\end{itemize}

\paragraph{}Here is an example proposition followed by an unnumbered remark:

\begin{defn}[Grothendieck's Prime]\label{defn:groth-prime}
	The Grothendieck's prime is the number $57$.
\end{defn}

\begin{rmrk*}
	The Grothendieck's prime is in fact not a prime. A random variable is not random nor a variable.
\end{rmrk*}

\section{Referencing}
\label{sect:referencing}

\paragraph{}One can use \code{Cref} (or \code{cref} if you don't want capitalization) for referencing environments created by \code{amstheorem} by their labels or sections, chapters, and use \code{cite} to cite bibliographical entries.

\begin{exmp*}[Referencing Example]
	Here are some references generated by the command \code{Cref}: \Cref{chap:placeholder}, \Cref{sect:dir-struct}, \Cref{defn:groth-prime}. Here is a wonderful book by Bersekas: \cite{bertsekas2009convex}.
\end{exmp*}

\paragraph{}Note that the citing style is now specified as \code{ieee}; see the options for \code{biblatex} package.


